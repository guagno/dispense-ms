%%%%%%%%%%%%%%%%%%%%%%%%%%%%%%%%%%%%%%%%%%%%%%%%%%%%%%%%%%%%%%%%%%%%%%%%
\chapter{Appendice matemarica}
\label{app:matematica}
%%%%%%%%%%%%%%%%%%%%%%%%%%%%%%%%%%%%%%%%%%%%%%%%%%%%%%%%%%%%%%%%%%%%%%%%

%%%%%%%%%%%%%%%%%%%%%%%%%%%%%%%%%%%%%%%%%%%%%%%%%%%%%%%%%%%%%%%%%%%%%%%%
\section{Alcune funzioni e formule utili}
%%%%%%%%%%%%%%%%%%%%%%%%%%%%%%%%%%%%%%%%%%%%%%%%%%%%%%%%%%%%%%%%%%%%%%%%

%%%%%%%%%%%%%%%%%%%%%%%%%%%%%%%%%%%%%%%%%%%%%%%%%%%%%%%%%%%%%%%%%%%%%%%%
\subsection{La funzione $\Gamma$}
\label{subsec:app1-gamma}
%%%%%%%%%%%%%%%%%%%%%%%%%%%%%%%%%%%%%%%%%%%%%%%%%%%%%%%%%%%%%%%%%%%%%%%%

La funzione $\Gamma$ di Eulero è definita da:
\be
\label{eq:defGamma}
\Gamma(\nu) = \int_0^\infty\de{x} \, e^{-x} \, x^{\nu-1} \quad\quad \nu > 0
\ee
Notiamo subito che
\be
\Gamma(1) = 1
\ee
Integrando per parti la (\ref{eq:defGamma}) otteniamo l'equazione di ricorrenza
\be
\Gamma(\nu+1) = \nu \Gamma(\nu)
\ee
Dalla precedente ricaviamo subito
\be
\Gamma(\nu+1) = \nu(\nu-1)(\nu-2) \, \cdots \, (1+p) p \, \Gamma(p) \quad\quad 0 < p \le 1
\ee
nella quale $p$ è la parte frazionaria di $\nu$. Se $\nu$ è intero abbiamo l'identificazione
\be
\Gamma(n+1) = n!
\ee
mentre se è semintero otteniamo
\bea
\label{eq:GammaSemint}
\Gamma \left( m + \dfrac{1}{2} \right) &\equiv& \left( m - \dfrac{1}{2} \right)! =
\left( m - \dfrac{1}{2} \right) \left( m - \dfrac{3}{2} \right) \, \cdots \, 
\left( \dfrac{1}{2} \right) \Gamma\left( \dfrac{1}{2} \right) \nonumber \\
&=& \dfrac{(2m-1)(2m-3) \, \cdots \, 1}{2^m}\pi^{1/2}
\eea
nella quale abbiamo sfruttato il fatto che
\be
\label{eq:Gamma12}
\Gamma \left( \dfrac{1}{2} \right) = \pi^{1/2}
\ee

\noindent
Cambiando variabile, $x = \alpha y^2$, possiamo scrivere la (\ref{eq:defGamma}) come
\be
\Gamma(\nu) = 2\alpha^\nu \int_0^\infty \de{y} \, e^{-\alpha y^2} \, y^{2\nu-1}
\ee
e ponendo $\mu = 2\nu - 1$ possiamo calcolare gli integrali notevoli
\be
I_\mu(\alpha) \equiv \int_0^\infty \de{y} \, e^{-\alpha y^2} \, y^{\mu}
= \dfrac{1}{2\alpha^{(\mu+1)/2}}\Gamma \left( \dfrac{\mu+1}{2} \right)
\ee

\noindent
Anche gli integrali appena definiti soddisfano una relazione di ricorrenza:
\be
I_{\mu+2}(\alpha) = -\dfrac{\text{d}}{\de{\alpha}} I_\mu(\alpha)
\ee
Per $\mu$ pari abbiamo, per esempio,
\be
I_0(\alpha) = \dfrac{1}{2}\left( \dfrac{\pi}{\alpha}   \right)^{1/2} \quad
I_2(\alpha) = \dfrac{1}{4}\left( \dfrac{\pi}{\alpha^3} \right)^{1/2} \quad
I_4(\alpha) = \dfrac{3}{8}\left( \dfrac{\pi}{\alpha^5} \right)^{1/2}
\ee
mentre per $\mu$ dispari
\be
I_1(\alpha) = \dfrac{1}{2\alpha}   \quad
I_3(\alpha) = \dfrac{1}{2\alpha^2} \quad
I_5(\alpha) = \dfrac{1}{\alpha^3}
\ee
Notiamo infine che abbiamo
\be
\int_{-\infty}^\infty \de{y} \, e^{-\alpha y^2} \, y^{\mu} = 
\left\{
\begin{array}{ll}
2I_\mu(\alpha) & \quad \text{se\ } \mu \text{\ è pari} \\
0              & \quad \text{altrimenti}
\end{array}
\right.
\ee

%%%%%%%%%%%%%%%%%%%%%%%%%%%%%%%%%%%%%%%%%%%%%%%%%%%%%%%%%%%%%%%%%%%%%%%%
\subsection{La formula di Stirling}
\label{subsec:app1-stirling}
%%%%%%%%%%%%%%%%%%%%%%%%%%%%%%%%%%%%%%%%%%%%%%%%%%%%%%%%%%%%%%%%%%%%%%%%

La formula di Stirling può essere ottenuta applicando la formula di Eulero--Maclaurin. Per definizione abbiamo
\be
\ln\left(\, n! \,\right) = \sum_{k=1}^{n} \ln k
\ee
e sostituendo l'integrale alla somma possiamo scrivere
\be
\ln\left(\, n! \,\right) \simeq \int_1^n \de{x} \, \ln x
= \big[x\ln x - x \big]_1^n \simeq n\ln n - n
\ee

%%%%%%%%%%%%%%%%%%%%%%%%%%%%%%%%%%%%%%%%%%%%%%%%%%%%%%%%%%%%%%%%%%%%%%%%
\subsection{Il teorema multinomiale}
\label{subsec:app1-multinomiale}
%%%%%%%%%%%%%%%%%%%%%%%%%%%%%%%%%%%%%%%%%%%%%%%%%%%%%%%%%%%%%%%%%%%%%%%%

È un'estensione del teorema binomiale. Dato un set di interi $\seti{k_i}$ con $i = 1, 2, \dots, n$ che soddisfano il vincolo
\be
\sum_{i=1}^n k_i = N
\ee 
si può dimostrare che
\be
\sum_{\seti{k_i}}\left[
N!\prod_{i=1}^n \dfrac{x_i^{k_i}}{k_i!}
\right] = (x_1 + x_2 + \cdots + x_n)^N
\ee