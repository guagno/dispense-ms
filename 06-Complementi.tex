%%%%%%%%%%%%%%%%%%%%%%%%%%%%%%%%%%%%%%%%%%%%%%%%%%%%%%%%%%%%%%%%%%%%%%%%
\chapter{Sistemi interagenti: un'introduzione}
\label{cap:interagenti}
%%%%%%%%%%%%%%%%%%%%%%%%%%%%%%%%%%%%%%%%%%%%%%%%%%%%%%%%%%%%%%%%%%%%%%%%

%%%%%%%%%%%%%%%%%%%%%%%%%%%%%%%%%%%%%%%%%%%%%%%%%%%%%%%%%%%%%%%%%%%%%%%%
\section{La distribuzione di Maxwell delle velocità}
%%%%%%%%%%%%%%%%%%%%%%%%%%%%%%%%%%%%%%%%%%%%%%%%%%%%%%%%%%%%%%%%%%%%%%%%


%%%%%%%%%%%%%%%%%%%%%%%%%%%%%%%%%%%%%%%%%%%%%%%%%%%%%%%%%%%%%%%%%%%%%%%%
\subsection{La legge di Graham}
%%%%%%%%%%%%%%%%%%%%%%%%%%%%%%%%%%%%%%%%%%%%%%%%%%%%%%%%%%%%%%%%%%%%%%%%
Se il contenitore $V$ contiene due tipi diversi di molecole, di massa $m_{A}$ e $m_{B}$, si ottiene facilmente che i {\em rate} di effusione dei due gas sono diversi, e vale il rapporto
%
\be
\frac{R_{A}}{R_{B}} = \sqrt{\frac{m_{B}}{m_{A}}}
\ee
%
quindi il gas più leggero effonde più rapidamente.


Sotto queste condizioni possiamo immaginare di isolare, all'interno del nostro sistema, un sottosistema intorno alla coordinata $\myvec{r}$ tale che il potenziale $U(\myvec{r})$ possa essere considerato sostanzialmente costante all'interno del sottosistema, anche se il sottosistema stesso è da intendersi come macroscopico a tutti gli effetti pratici. Nel caso del gas in un cilindro possiamo pensare a un cilindretto posto a quota $y$, di altezza $\de{y}$ e la cui base sia parallela alla base del cilindro. All'interno di questo sottosistema possiamo considerare il potenziale gravitazionale come costante e pari a $mgy$.

Nel capitolo \ref{cap:canonico} abbiamo scoperto che la funzione di partizione di {\em singola particella} per un gas ideale classico è
\be
Q_1 = V/\lambda^3
\ee
in cui $\lambda$ è la lunghezza d'onda termica. Invece nel capitolo \ref{cap:grancanonico} abbiamo visto che il $q$--potenziale del sistema è
\be
q(z,V,T) = \ln\calQ(z,V,T) = zV/\lambda^3
\ee
in cui la $z = e^{\mu/kT}$. Per il gas ideale classico possiamo quindi scrivere
\be
q(z,V,T) = zQ_1
\ee 
D'altronde per il numero medio di particelle $N$ abbiamo
\be
\label{eq:conGC0}
N = z\left[ \dpar{}{z}q(z,V,T) \right]_{V,T} = zQ_1
\ee
Nessuno però può negarci il diritto di scrivere
\be
\label{eq:conGC1}
Q_1 = \sum_\varepsilon e^{-\beta\varepsilon}
\ee
in cui la somma è su tutti i livelli di energia di singola particella del sistema. Possiamo però anche scrivere
\be
\label{eq:conGC2}
N = \sum_\varepsilon \langle n_\varepsilon\rangle
\ee
nella quale $\langle n_\varepsilon\rangle$ rappresenta il valore d'aspettazione del numero di particelle sul livello d'energia $\varepsilon$.

Confrontando la (\ref{eq:conGC0}) con la (\ref{eq:conGC2}) e usando la (\ref{eq:conGC1}) otteniamo, per un gas ideale classico,
\be
\langle n_\varepsilon\rangle = e^{(\mu-\varepsilon)/kT}
\ee
Questo risultato, che andremo presto a utilizzare per l'obiettivo che ci siamo posti in questa sezione, risulterà utile anche in futuro, quando dovremo confrontare la stessa quantità, ma calcolata usando un formalismo completamente quantistico, quindi quando studieremo gas ideali di Bose--Einstein o di Fermi--Dirac.

La presenza di un potenziale esterno cambia le carte in tavola. Da una parte abbiamo che il potenziale chimico del sistema in equilibrio non può che essere costante, perché all'equilibrio non possiamo ammettere spostamenti netti di materia da una parte all'altra del sistema. Dall'altra, insistendo a chiamare $\varepsilon$ solo la parte cinetica dell'energia di una molecola, dobbiamo scrivere
\be
\langle n_\varepsilon\rangle = e^{[(\mu-mgy) - \varepsilon]/kT}
\ee
Possiamo dunque pensare al sistema con un campo esterno come a un sistema {\em senza} campo esterno ma con un {\em potenziale chimico effettivo}
\be
\mu_{\textrm{eff}}(y) = \mu - mgy
\ee
che dipende dalle coordinate. La variazione di $\mu_{\textrm{eff}}$ con $y$ è costruita in modo tale da generare una densità {\em dipendente dalla posizione} che mantiene il sistema all'equilibrio.

Per un gas ideale classico, la relazione che lega il potenziale chimico alla densità è
\be
\mu = kT\ln(n\lambda^3)
\ee
Nelle condizioni imposte dall'esistenza del potenziale gravitazionale, nelle quali la densità $n$ dipende necessariamente dalle coordinate, dobbiamo invece scrivere
\be
\mu_{\textrm{eff}}(y) = \mu - mgy = kT\ln(n(y)\lambda^3)
\ee
Poiché stiamo studiando un sottostistema (macroscopico) all'equilibrio a temperatura $T$, l'equazione di stato dei gas ideali continua a essere valida, e possiamo scrivere
\be
\mu - mgy = kT\ln\left(\dfrac{P(y)\lambda^3}{kT}\right)
\ee
e derivando rispetto a $y$, considerando che $\mu$ è constante, otteniamo
\be
\dfrac{\de{P(y)}}{P(y)} = -\dfrac{mg}{kT}\de{y}
\ee
che integrata ci dà facilmente
\be
P(y) = P(0)e^{-mgy/kT}
\ee
La precedente è la {\em formula barometrica dell'atmosfera isoterma}, ottenuta per la prima volta da Boltzmann (con un metodo completamente diverso).

Sostituendo $y = L$ in questa equazione e normalizzando correttamente (tutto ciò è lasciato come esercizio) otteniamo, per la pressione sulla faccia superiore del cilindro, la stessa espressione che avevamo ottenuto in precedenza, eq. (\ref{eq:PaL}).