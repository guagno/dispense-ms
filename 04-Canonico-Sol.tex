%%%%%%%%%%%%%%%%%%%%%%%%%%%%%%%%%%%%%%%%%%%%%%%%%%%%%%%%%%%%%%%%%%%%%%%%
\chapter{Soluzione degli esercizi del Capitolo \ref{cap:canonico}}
%%%%%%%%%%%%%%%%%%%%%%%%%%%%%%%%%%%%%%%%%%%%%%%%%%%%%%%%%%%%%%%%%%%%%%%%

%%%%%%%%%%%%%%%%%%%%%%%%%%%%%%%%%%%%%%%%%%%%%%%%%%%%%%%%%%%%%%%%%%%%%%%%
\section*{Esercizio \ref{ex:04-guc}}
\addcontentsline{toc}{section}{Esercizio \ref{ex:04-guc}}
%%%%%%%%%%%%%%%%%%%%%%%%%%%%%%%%%%%%%%%%%%%%%%%%%%%%%%%%%%%%%%%%%%%%%%%%
Scriviamo, come al solito per sistemi non interagenti,
\be
Q_N(V,T) = \frac{1}{N!}[Q_1(V,T)]^N
\ee
con $Q_1$ funzione di partizione di singola particella. Abbiamo
\be
Q_1(V,T) = \frac{1}{h^3}\int \de^3 q\int \de^3 p \,e^{-\beta pc}
\ee
e quindi possiamo calcolare immediatamente
\be
Q_1(V,T) = \frac{8\pi V}{(\beta hc)^3}
\ee
L'energia libera di Helmholtz è
\be
A(V,T) = -kT\ln Q_N(V,T) = -kTN
\left\{
\ln\left( \frac{8\pi V(kT)^3}{N(hc)^3} \right)
\right\}
\ee
Otteniamo subito
\be
\label{eq:pguc}
P = -\dparc{A}{V}{N,T} = \frac{NkT}{V}
\ee
ovvero l'equazione di stato (identica a quella del gas non relativistico). Per l'energia abbiamo invece
\be
\label{eq:uguc}
U = -\dpar{\ln Q_N}{\beta} = 3NkT
\ee
La relazione tra pressione, volume ed energia è quindi $PV = U/3$. Passando all'entropia, abbiamo
\be
S = \frac{U-A}{T}
\ee
e con pochi passaggi otteniamo subito
\be
S = Nk
\left\{
\ln\left( \frac{8\pi V(kT)^3}{N(hc)^3} \right) + 4
\right\}
\ee
Se vogliamo confrontarci con il risultato microcanonico, cioè con l'eq. (\ref{eq:sgum}), dobbiamo inserire nella precedente l'energia interna; dalla (\ref{eq:uguc}) otteniamo subito
\be
S(U,N,V) = Nk\left\{\ln\left(\frac{8\pi VU^3}{27(hc)^3 N^4}\right) + 4\right\}
\ee
Se consideriamo $S = \mathrm{costante}$, allora abbiamo l'implicazione $VU^3 = \mathrm{costante}$ (per $N = \mathrm{costante}$), e otteniamo subito, usando la (\ref{eq:pguc}), il rapporto $C_P/C_V$:
\be
PV^{4/3} = \mathrm{costante} \quad
\gamma \equiv C_P/C_V = 4/3
\ee
Per la densità degli stati abbiamo
\be
g(E) = \frac{1}{2\pi i}\oint e^{\beta E}Q(\beta)\de\beta
\ee
in cui abbiamo già previsto di chiudere l'integrale di linea sul piano complesso nella direzione $\beta\to -\infty$. Si trova facilmente
\be
g(E) = \frac{1}{N!}
\left\{
\frac{8\pi V}{(hc)^3}
\right\}^N \frac{1}{2\pi i}\oint \frac{e^{\beta E}}{\beta^{3N}} \simeq
\frac{1}{N!}
\left\{
\frac{8\pi VE^3}{(hc)^3}
\right\}^N
\ee
in cui nell'ultimo passaggio abbiamo usato $3N-1 \simeq 3N$ (limite termodinamico).

%%%%%%%%%%%%%%%%%%%%%%%%%%%%%%%%%%%%%%%%%%%%%%%%%%%%%%%%%%%%%%%%%%%%%%%%
\section*{Esercizio \ref{ex:04-oscfermi}}
\addcontentsline{toc}{section}{Esercizio \ref{ex:04-oscfermi}}
%%%%%%%%%%%%%%%%%%%%%%%%%%%%%%%%%%%%%%%%%%%%%%%%%%%%%%%%%%%%%%%%%%%%%%%%
La funzione di partizione di singola particella è:
\be
Q_1 = e^{-\beta\varepsilon} + e^{\beta\varepsilon} = 2 \cosh(\beta\varepsilon)
\ee
Troviamo subito il numero di occupazione per i due livelli:
\bea
N_{-} &=& N \dfrac{e^{\beta\varepsilon}}{e^{-\beta\varepsilon} + e^{\beta\varepsilon}}
= \dfrac{N}{1 + e^{-2\beta\varepsilon}} \nonumber \\
N_{+} &=& N \dfrac{e^{-\beta\varepsilon}}{e^{-\beta\varepsilon} + e^{\beta\varepsilon}}
= \dfrac{N}{1 + e^{2\beta\varepsilon}}
\eea

%%%%%%%%%%%%%%%%%%%%%%%%%%%%%%%%%%%%%%%%%%%%%%%%%%%%%%%%%%%%%%%%%%%%%%%%
\section*{Esercizio \ref{ex:04-eqchim}}
\addcontentsline{toc}{section}{Esercizio \ref{ex:04-eqchim}}
%%%%%%%%%%%%%%%%%%%%%%%%%%%%%%%%%%%%%%%%%%%%%%%%%%%%%%%%%%%%%%%%%%%%%%%%

All'equilibrio termodinamico i potenziali chimici devono essere uguali da entrambi i lati della reazione, quindi
\be
	\label{eq:sol04-mueq}
	\mu_a + \mu_b = \mu_{ab}
\ee
	Sappiamo che 
\be
	\mu_x = \dparc{A_x}{N_x}{T,V}= -kT\dparc{\ln Q(N_x)}{N_x}{T,V}
\ee
	La funzione di partizione per un gas ideale classico è data da
\be
	Q(N_x) = \dfrac{1}{N_x!}\left[Q_x\right]^{N_x}
\ee
	quindi, usando la formula di Stirling, 
\be
	\mu_x = -kT\dparu{N_x}\left[ N_x\ln Q_x - N_x\ln N_x + N_x \right] = -kT[\ln Q_x - \ln N_x]
\ee
	Sostituendo nella (\ref{eq:sol04-mueq}) e ricordando che $N_x = n_x V$ si ottiene facilmente
\be 
	\ln{\dfrac{Q_a Q_b}{Q_{ab}}}= \ln \dfrac{N_a N_b}{N_{ab}} \quad\rightarrow\quad 
	\dfrac{n_{ab}}{n_a n_b} = V\dfrac{Q_{ab}}{Q_{a}Q_{b}}
\ee

%%%%%%%%%%%%%%%%%%%%%%%%%%%%%%%%%%%%%%%%%%%%%%%%%%%%%%%%%%%%%%%%%%%%%%%%
\section*{Esercizio \ref{ex:04-esps}}
\addcontentsline{toc}{section}{Esercizio \ref{ex:04-esps}}
%%%%%%%%%%%%%%%%%%%%%%%%%%%%%%%%%%%%%%%%%%%%%%%%%%%%%%%%%%%%%%%%%%%%%%%%
Abbiamo visto  che l'entropia di un gas ideale classico vale
\be
\frac{S}{Nk} = \ln\left(\frac{V}{N\lambda^3}\right) + \frac{5}{2}
\ee
in cui $\lambda$ è la lunghezza d'onda termica (o di de Broglie). Ai fini dell'esercizio ci basta ricordare che
\be
\lambda^3 = cT^{-3/2}
\ee
con $c$ indipendente dalla temperatura $T$,  per cui ricordando che $Q_1 = V/\lambda^3$ possiamo scrivere
\be
\ln\left(\frac{Q_1}{N}\right) + T\dparc{\ln Q_1}{T}{P} =
\ln\left(\frac{V}{N\lambda^3}\right) + 3/2 + \frac{T}{V}\dparc{V}{T}{P}
\ee
Per un gas ideale abbiamo la relazione $V = NkT/P$, quindi
\be
\frac{T}{V}\dparc{V}{T}{P} = \frac{NkT}{PV} = 1
\ee
da cui discende ciò che dovevamo dimostrare.